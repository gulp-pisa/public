\documentclass{beamer}

\beamertemplateshadingbackground{red!10}{blue!10} %10 10

\usetheme{Warsaw}

\beamertemplatetransparentcovereddynamic

\usepackage[utf8x]{inputenc}
\usepackage[T1]{fontenc}

\usepackage[italian]{babel}

\title{Web Radio}
\author{Gabriele ``LightKnight'' Stilli}
\institute{G.U.L.P. -- Gruppo Utenti Linux Pisa}
\date{Linux Day, Pisa, 27/10/2007}

\begin{document}
\frame{\titlepage}

\frame{\tableofcontents}

\section{Come mai si crea una Web Radio}

\begin{frame}\frametitle{Web Radio: come mai?}
\begin{itemize}
\item Per condividere la propria musica con gli altri
\item Per fare la cronaca in diretta di eventi
\item Per scambiarsi file: alternativa Lo-Fi al P2P
\item Per divertirsi a fare il DJ
\item {\ldots}
\end{itemize}
\end{frame}

\section{Struttura di una Web Radio}

\subsection{La catena dei moduli}

\begin{frame}\frametitle{Molti programmi, uno scopo}
Una Web Radio è composta da più programmi disposti in sequenza;
ciascuno ha per input l'output del programma precedente.

Questo perché la struttura modulare si presta meglio ad essere
adattata a diverse esigenze, ad esempio miscelare altri stream da
fonti esterne.

Ciò riflette anche una tipica tradizione Unix: è meglio avere più
programmi separati che fanno ciascuno una cosa che un unico programma
tuttofare.

\end{frame}

\begin{frame}\frametitle{Elenco dei moduli}
Nello specifico, una semplice Web Radio è composta da:
\begin{description}
\item[Player] che riproduce i file audio e genera il flusso di dati
grezzo
\item[Encoder] che codifica il flusso grezzo oppure legge i file da
una playlist e li riproduce
\item[Server] che riceve il flusso codificato e lo invia in rete
\end{description}

Inoltre, il PC su cui opererà la Web Radio deve avere un sistema audio
già funzionante, possibilmente utilizzando i moduli ALSA.
\end{frame}

\subsection{Il mixer e la sua configurazione}

\begin{frame}\frametitle{Configurazione del mixer}
Il mixer va configurato in modo da selezionare il canale corretto per
la cattura audio. ``Mix'' dovrebbe essere una buona scelta, ma dipende
anche dalla scheda audio specifica.

Inoltre, è essenziale regolare con attenzione i volumi dei canali
audio interessati e dell'eventuale microfono.
\end{frame}

\subsection{Il player}

\begin{frame}\frametitle{Il player}
È possibile utilizzare qualunque programma per riprodurre file audio;
basta configurarlo in modo che il suo canale di output sia l'input
dell'encoder.
\end{frame}

\subsection{L'encoder}

\begin{frame}\frametitle{Funzione dell'encoder}
L'encoder ha il compito di raccogliere lo stream audio generato dal
player e codificarlo nel formato preferito, da inviare poi allo
streaming server. È qui che si decide il formato (o i formati) da
adottare per la Web Radio.

Alcuni encoder possono generare lo stream audio anche leggendo dei
file da una playlist. Ovviamente i file devono essere in un formato
riconoscibile dall'encoder.

Vediamo la configurazione di ices2, che permette di generare stream
MP3 e Vorbis.
\end{frame}

\begin{frame}[containsverbatim,squeeze]
\frametitle{Configurazione di ices2: input da player}
\begin{verbatim}
<input>
    <module>alsa</module>
    <param name="rate">48000</param>
    <param name="channels">2</param>
    <param name="device">hw:0,0</param>
    <param name="metadata">1</param>
    <param name="metadatafilename">test</param>
</input>
\end{verbatim}
\end{frame}

\begin{frame}[containsverbatim,squeeze]
\frametitle{Configurazione di ices2: input da playlist}
\begin{verbatim}
<input>
    <module>playlist</module>
    <param name="type">basic</param>
    <param name="file">playlist.txt</param>
    <param name="random">0</param>
    <param name="restart-after-reread">0</param>
    <param name="once">0</param>
</input>
\end{verbatim}
\end{frame}

\begin{frame}[containsverbatim,squeeze]
\frametitle{Configurazione di ices2: output sul server}
\footnotesize
\begin{verbatim}
<instance>
    <hostname>localhost</hostname>
    <port>8000</port>
    <password>password</password>
    <mount>/example1.ogg</mount>
    <encode>
        <quality>0</quality>
        <samplerate>48000</samplerate>
        <channels>2</channels>
    </encode>
    <resample>
        <in-rate>48000</in-rate>
        <out-rate>24000</out-rate>
    </resample>
</instance>
\end{verbatim}
\end{frame}

\subsection{Lo streaming server}

\begin{frame}\frametitle{Funzione dello streaming server}
È la parte della catena esposta alla rete.

Si occupa di ricevere gli stream audio, riunirli in uno solo ed
inviare quest'ultimo verso l'esterno.

Può gestire più stream audio facendoli uscire su porte differenti,
nonché permettere il relay dei propri stream su altri server.

Uno dei più diffusi è icecast2, che fra l'altro può anche essere
controllato da remoto tramite interfaccia web.
\end{frame}

\begin{frame}[containsverbatim,squeeze]
\frametitle{Configurazione di icecast2}
\footnotesize
\begin{verbatim}
<authentication>
    <source-password>source-password</source-password>
    <relay-password>relay-password</relay-password>

    <admin-user>admin</admin-user>
    <admin-password>admin-password</admin-password>
</authentication>

<relay>
    <server>127.0.0.1</server>
    <port>8001</port>
    <mount>/example.ogg</mount>
    <local-mount>/different.ogg</local-mount>
</relay>
\end{verbatim}
\end{frame}

\subsection{Firewall e port forwarding}

\begin{frame}\frametitle{Firewall e port forwarding}
Per poter trasmettere lo stream, occorre che la porta su cui esce lo
streaming server sia visibile alla rete. Ciò vuol dire che, nel caso
sia presente un firewall fra la macchina e la rete, la suddetta porta
deve essere aperta.

Se inoltre la macchina è in LAN o comunque accede alla rete tramite un
router, è necessario effettuare il forwarding della porta dal router
alla macchina server.
\end{frame}

\subsection{Cenni sui formati}

\begin{frame}\frametitle{Cenni sui formati}
Una Web Radio può trasmettere in svariati formati. I principali sono
Ogg Vorbis, MP3, WMA, AAC.

Ogg Vorbis è un formato aperto, con specifiche pubbliche, non gravato
da brevetti e di qualità confrontabile con gli altri, quando non
leggermente superiore.
\end{frame}

\section{Il client}

\begin{frame}\frametitle{Il client}
Per ascoltare una web radio che utilizzi Icecast, si può usare un
qualunque riproduttore software che supporti il protocollo HTTP oltre,
ovviamente, al formato audio scelto per lo stream.

Alcuni esempi: XMMS (e derivati, ad es. Audacious), Mplayer, Amarok,
VLC, Xine, Totem, Rhythmbox, Kaffeine, Exaile, Somaplayer{\ldots}
\end{frame}

\section{Programmi di gestione}

\begin{frame}\frametitle{Introduzione}
Quanto abbiamo visto finora è sufficiente per configurare una Web
Radio a livello base.

Ma se vogliamo cambiare il palinsesto, dobbiamo riavviare il player o
modificare a mano la playlist; questo, a lungo andare, diventa
meccanico e noioso.

Qui di seguito, vediamo alcuni programmi che ci aiutano a gestire il
palinsesto della nostra radio.
\end{frame}

\begin{frame}\frametitle{Presentazione di SomaSuite}
SomaSuite è una suite di programmi per gestire la programmazione di
una radio. Grazie ad essa è possibile creare palinsesti, programmare
l'esecuzione di determinati brani ad una certa ora, inserire jingle o
spot fra un brano e l'altro e, in generale, avere un buon controllo su
cosa e quando si trasmette.

Fornisce inoltre librerie per sviluppare programmi di controllo in
vari linguaggi (C/C++, PHP, Python{\ldots}).
\end{frame}

\begin{frame}\frametitle{Somad}
Somad è il programma principale della SomaSuite. È un demone che si
occupa di gestire i file di palinsesto e di spot.

A seconda delle direttive scritte in questi file, può lanciare il
player audio per riprodurre il brano indicato, o una serie di brani in
modalità random, o ritrasmettere programmazioni precedenti; può
trasmettere spot a orari prestabiliti o a cadenze predefinite.

Può essere controllato tramite librerie (LibSoma, PySoma, PhpSoma),
programmi grafici (SomaX) o testuali (SomaClient).
\end{frame}

\begin{frame}[containsverbatim,squeeze]
\frametitle{Configurazione di somad}
\begin{verbatim}
User = soma 
Group = users
Background = false
ServerName = 127.0.0.1
Port = 12521
Password = ``passwd''
ProgramItem = ``mplayer''
OptionsItem = ``-ao alsa''
Palinsesto = ``/etc/somad/palinsesto.cfg''
Spot = ``/etc/somad/spot.cfg''
\end{verbatim}
\end{frame}

\begin{frame}\frametitle{Somax}
SomaX è un client grafico completo per il controllo di Somad. Permette
di creare graficamente o in modo testuale i file di palinsesto,
modificarli dinamicamente e in tempo reale e salvarli, nonché di
visualizzare la cronologia della radio, sia passata che futura, ed
agire su di essa.
\end{frame}

\begin{frame}\frametitle{Altri componenti della SomaSuite (1)}
\begin{description}
\item[SomaPlayer] È il player audio della SomaSuite, che funge anche
da encoder. Può ricevere in input una qualunque sorgente audio (file,
stream, microfono, line in) e restituire in output lo stream già
formato.
\item[SomaMDD] Legge i metadati dai file audio e li passa allo
streaming server; in questo modo i player compatibili possono mostrare
le informazioni sul brano e sullo stream.
\item[SdS] È il mixer della SomaSuite, che serve a mixare più sorgenti
audio differenti ed inviarle allo streaming server.
\end{description}
\end{frame}

\begin{frame}\frametitle{Altri componenti della SomaSuite (2)}
\begin{description}
\item[SomaMysql] Modulo per supportare interrogazioni MySQL per la
selezione dei brani da mandare in onda.
\item[SomaRss] Modulo che interroga pagine RSS 0.91 e esegue il
contenuto dei tag link.
\item[SomaRun] Modulo che esegue un comando.
\item[SomaHttp] Modulo per mandare in onda il contenuto di directory
list su http (Index Of).
\end{description}
\end{frame}

\section{Problemi di banda}

\begin{frame}\frametitle{La banda in upstream}
Uno dei problemi più grandi del gestire una Web Radio da casa propria
è la banda in upstream, senza la quale qualunque trasmissione è
impossibile.

Purtroppo, per una ADSL casalinga, la banda in upstream a disposizione
è veramente mediocre: di solito 256kbps (32 Kb/s), con punte di 1 Mbps
per pochi fortunati.

Occorre quindi trovare un espediente per alleggerire il carico sul
server, senza pregiudicare qualità e accessibilità.
\end{frame}

\begin{frame}\frametitle{Peer to peer e Web Radio}
È possibile applicare i concetti di base del peer to peer alle Web
Radio.

Anziché prelevare lo stream da un unico server, ogni client si fa
carico di una parte di banda in uscita, ritrasmettendo lo stream
ricevuto e così diventando server a sua volta.

In questo modo, il server deve sopportare il carico di un numero
minore di client.

Possibili problemi:
\begin{itemize}
\item protocollo più complicato rispetto al normale HTTP
\item necessità di gestire lo stream in tempo reale
\item in particolare, gestire la connessione/disconnessione dei client
\end{itemize}
\end{frame}

\begin{frame}\frametitle{PeerCast}
PeerCast è un sistema client-server che sfrutta le proprietà del peer
to peer per trasmettere il proprio stream minimizzando l'utilizzo di
banda.

È possibile farsi pubblicare su un apposito servizio di Yellow Pages;
in questo modo sarà più facile farsi trovare, ma è anche possibile
generare i link peercast necessari per sfruttare la modalità peer to
peer.

È possibile inoltre configurarlo per accettare un certo numero di
client che si connettano in modalità ``normale''.
\end{frame}

\begin{frame}\frametitle{Configurazione server e client}
Il server PeerCast ha una configurazione molto simile ad uno streaming
server tradizionale, con poche opzioni peculiari (indirizzo del server
Yellow Pages, limitazioni alle connessioni dirette, ecc.).

I client che volessero sfruttare la modalità peer to peer necessitano
di un client che riceva un URL in formato peercast e, consultando il
servizio di Yellow Pages, permettano ad un player audio di collegarsi
ad uno dei peer.

Altrimenti, per le stazioni che lo permettano, è possibile collegarsi
direttamente al server, come con una radio basata su Icecast.
\end{frame}

\section{Cenni normativi}

\begin{frame}\frametitle{Cosa dice la legge(?)}
Per poter trasmettere in rete bisogna anche essere in regola con la
legge sul Diritto d'Autore e le normative sulle trasmissioni, che
coprono le radio e televisioni tradizionali, ma anche le Web Radio.

In particolare, è necessario stabilire se e quali autorizzazioni si
debbano ottenere dalla SIAE e quali tariffe vanno ad essa pagate.

(Quanto segue non costituisce parere legale, ma è esclusivamente
l'opinione di chi scrive, al meglio delle sue conoscenze)
\end{frame}

\begin{frame}\frametitle{Normative SIAE}
In caso si vogliano trasmettere brani musicali registrati alla SIAE o
altri enti di raccolta diritti analoghi operanti all'estero, è
necessario compilare un apposito modulo di Autorizzazione per la Web
Radio (AWR) e pagare la rispettiva tariffa.

La normativa è complessa e restrittiva: non è possibile far salvare su
disco lo stream audio, né far scegliere all'ascoltatore quali brani
ascoltare. Inoltre, non è possibile ritrasmettere lo stream (questo
include sistemi quali PeerCast).

La tariffa è differenziata a seconda che la radio sia personale,
istituzionale o commerciale e a seconda della quantità di ascoltatori
(più o meno di 30). L'importo varia dai 240 euro all’anno per radio
personali, a cifre molto più alte per radio commerciali.
\end{frame}

\begin{frame}\frametitle{Musica libera}
Se invece si trasmette solo musica non registrata alla SIAE, allora
non c'è bisogno di alcuna autorizzazione. In particolare, la musica
rilasciata sotto licenze libere (Creative Commons, LAL, GPL, altre)
non necessita di autorizzazioni SIAE.

\begin{itemize}
\item Jamendo;
\item The Internet Archive;
\item Siti nazionali di Creative Commons;
\item {\ldots} e centinaia di netlabel sparse per la rete.
\end{itemize}

Consiglio della casa: \bf{trasmettete musica libera! :-)}
\end{frame}

\section{Note finali}

\begin{frame}\frametitle{Link utili}
Icecast: http://icecast.org/

Peercast: http://peercast.org/

SomaSuite: http://somasuite.org/

Jamendo: http://jamendo.com/

The Internet Archive: http://archive.org/

Creative Commons: http://creativecommons.org/

Elenco di netlabel: http://netlabels.org/

Fabry Mondo: http://fabrymondo.wordpress.com/
\end{frame}

\begin{frame}\frametitle{Copyright (o Copyleft)}
Quest'opera è stata rilasciata sotto la licenza Creative Commons
Attribuzione-Condividi allo stesso modo 3.0 Unported. Per leggere una
copia della licenza visita il sito web
http://creativecommons.org/licenses/by-sa/3.0/ o spedisci una lettera
a Creative Commons, 171 Second Street, Suite 300, San Francisco,
California, 94105, USA.
\end{frame}

\end{document}
