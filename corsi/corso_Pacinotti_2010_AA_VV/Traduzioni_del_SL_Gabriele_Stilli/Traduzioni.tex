\documentclass{beamer}

\beamertemplateshadingbackground{red!10}{blue!10} %10 10

\usetheme{Warsaw}

\beamertemplatetransparentcovereddynamic

\usepackage[utf8x]{inputenc}
\usepackage[T1]{fontenc}

\usepackage[italian]{babel}

\title{Traduzioni di Software Libero}
\subtitle{Contribuire alla comunità del Software Libero senza essere
programmatori}
\author{Gabriele ``LightKnight'' Stilli}
\institute{G.U.L.P. -- Gruppo Utenti Linux Pisa}
\date{Pisa, 24/10/2009}

\begin{document}
\frame{\titlepage}

\frame{\tableofcontents}

\section{Perché tradurre?}

\begin{frame}\frametitle{Perché NON tradurre?}
\begin{itemize}
\item Le traduzioni sono brutte/sbagliate
\item Le traduzioni sono incomplete/vecchie
\item Tutti usano i programmi in inglese
\end{itemize}
\end{frame}

\begin{frame}\frametitle{Perché tradurre?}
\begin{itemize}
\item È utile alla comunità
\item Non c'è bisogno di essere programmatori
\item Abbassa una barriera d'ingresso al Software Libero
\item Può essere piacevole
\end{itemize}
\pause
Il dibattito è ancora in corso{\ldots} e mai finirà.
\end{frame}

\section{Tradurre un programma}

\begin{frame}\frametitle{Una lingua una copia?}
In passato i programmi erano monolingua; le traduzioni erano inglobate
nel programma.

Questo implicava che per avere il programma tradotto in un'altra
lingua se ne doveva acquisire un'altra copia.
\end{frame}

\begin{frame}\frametitle{Si può fare di meglio}
Oggi la traduzione di un programma viene posta in un file esterno,
fornito insieme al programma, che contiene le corrispondenze fra
stringhe originali e stringhe tradotte e a cui il programma stesso
accede tramite funzioni di libreria speciali.

Ciò implica che un programma per essere tradotto ha bisogno di essere
opportunamente preparato.
\end{frame}

\begin{frame}\frametitle{I18n e l10n}
\begin{center}
\textbf{I18n? L10n?}
\end{center}
\pause

\begin{description}
\item[Internationalization] È il procedimento con cui si inseriscono
nel programma le funzioni e i segnalibri necessari a tradurlo.
\pause
\item[Localization] È il procedimento di traduzione vero e proprio.
\end{description}
\pause

\begin{center}
\textbf{Non c'è l10n senza i18n!}
\end{center}
\end{frame}

\section{La suite gettext}

\begin{frame}\frametitle{Introduzione a gettext}
La suite gettext è un insieme di programmi che permettono a un
programma di essere traducibile.

Ciò si ottiene tramite funzioni che permettono la creazione di un
database di stringhe traducibili e delle loro controparti tradotte.
\end{frame}

\begin{frame}[containsverbatim]
\frametitle{Preparazione del sorgente (1)}
Prima della cura{\ldots} 

\begin{verbatim}

#include <stdio.h>
int main ()
{
    printf ("Hello world\n");
}

\end{verbatim}
\end{frame}

\begin{frame}[containsverbatim,squeeze]
\frametitle{Preparazione del sorgente (2)}
{\ldots} dopo la cura!
\begin{verbatim}
#include <stdio.h>
#include <libintl.h>
#include <locale.h>
#define PACKAGE "ciao"
#define LOCALEDIR "/var/tmp"

int main ()
{
    setlocale (LC_ALL, "");
    bindtextdomain (PACKAGE, LOCALEDIR);
    textdomain (PACKAGE);
    printf (gettext ("Hello world\n"));
}
\end{verbatim}
\end{frame}

\begin{frame}\frametitle{Dal sorgente alle stringhe}

Dal sorgente siffatto tramite xgettext viene creato il file .po
(Portable Object), che contiene l'elenco delle stringhe traducibili.

Dopo la traduzione, msgfmt si occuperà di compilare il file .po,
ottenendo un file .mo (Machine Object) che verrà utilizzato dal
programma.
\end{frame}

\begin{frame}\frametitle{Struttura di un file .po}
\begin{description}
\item [L'header] in cui sono scritti i dati del programma e della
traduzione;
\item [Le stringhe] il cuore della traduzione
\begin{itemize}
\item Stringhe tradotte, fuzzy, non tradotte, obsolete
\item Commenti alle stringhe
\item Gestione dei plurali e delle inversioni
\end{itemize}
\end{description}
\end{frame}

\begin{frame}[containsverbatim,squeeze]
\frametitle{L'header}
\begin{verbatim}
"Project-Id-Version: ciao-0.1\n"
"Report-Msgid-Bugs-To: \n"
"POT-Creation-Date: 2009-06-24 17:45+0200\n"
"PO-Revision-Date: 2009-06-24 17:58+0200\n"
"Last-Translator: Full Name <email@address>\n"
"Language-Team: My Language <LL@Li.org>\n"
"MIME-Version: 1.0\n"
"Content-Type: text/plain; charset=iso-8859-1\n"
"Content-Transfer-Encoding: 8bit\n"
"Plural-Forms: nplurals=2; plural=(n != 1)\n"
\end{verbatim}
\end{frame}

\begin{frame}[containsverbatim,squeeze]
\frametitle{Le stringhe: tradotte, fuzzy, obsolete}
\begin{verbatim}
#: pippo.c:14
#, c-format
msgid "Hello world\n"
msgstr "Ciao mondo\n"

#: mattricks/wxgui/matchstats/teamgrade.py:104
#, fuzzy
msgid "Grade"
msgstr "Partite"

#~ msgid "The match will begin in %s."
#~ msgstr "La partita inizierà fra %s."
\end{verbatim}
\end{frame}

\begin{frame}[containsverbatim,squeeze]
\frametitle{La gestione dei plurali}
\begin{verbatim}
#: mattricks/Commandline.py:742
#, python-format
msgid "There are %d element in the database."
msgid_plural "There are %d elements in the database."
msgstr[0] "C'è %d elemento nel database."
msgstr[1] "Ci sono %d elementi nel database."
\end{verbatim}
\end{frame}

\begin{frame}\frametitle{Cosa ci vuole per tradurre}

\begin{itemize}
\item Discreta conoscenza dell'inglese
\item Buona conoscenza dell'italiano
\item Una certa competenza della materia
\item Tempo libero e dedizione
\end{itemize}
\end{frame}

\begin{frame}\frametitle{Regole di buona traduzione}
\begin{itemize}
\item Rispetto dell'ortografia (accenti, maiuscole/minuscole,
spaziature{\ldots})
\item Forma impersonale
\item Tradurre, non spiegare
\item Eleganza
\item Coerenza
\item Autosufficienza
\item Riconoscibilità
\item [{\ldots}]
\end{itemize}
\end{frame}

\begin{frame}\frametitle{Strumenti di traduzione}
\begin{itemize}
\item po-mode (vari editor)
\item poedit
\item lokalize
\item freespeak
\item Rosetta
\item OmegaT
\item [{\ldots}]
\end{itemize}
\end{frame}

\begin{frame}\frametitle{E dopo la traduzione?}
Una volta tradotto il programma, è opportuno sottoporre il lavoro alla
comunità per la revisione. In questa fase si correggono gli errori e
si discutono le scelte stilistiche o di glossario da applicare.

Per i programmi, generalmente la comunità di riferimento è il
Translation Project: tp@lists.linux.it. Programmi o suite particolari
possono però avere le proprie comunità (GNOME, KDE,
distribuzioni{\ldots}).

Dopo la revisione, la traduzione può essere inviata allo sviluppatore
per la sua inclusione nel programma.
\end{frame}

\begin{frame}\frametitle{Programmi e non solo}
L'universo della traduzione è multiforme e sono tante le cose in cui
c'è bisogno d'aiuto:

\begin{itemize}
\item Programmi
\item Documentazione
\item Siti web
\item Descrizioni di pacchetti
\item po-debconf
\item [{\ldots}]
\end{itemize}
\end{frame}

\begin{frame}\frametitle{Link utili}
Italian Translation Project: http://tp.linux.it/

Gettext: http://www.gnu.org/software/gettext/

GNOME: http://www.it.gnome.org/index.php/Traduzioni

KDE: http://kde.gulp.linux.it/

Rosetta: https://launchpad.net/rosetta

OmegaT: http://www.omegat.org/it/omegat.html

Open-Tran: http://open-tran.eu/
\end{frame}

\begin{frame}\frametitle{Copyright (o Copyleft)}
Quest'opera è stata rilasciata sotto la licenza Creative Commons
Attribuzione-Condividi allo stesso modo 3.0 Unported. Per leggere una
copia della licenza visita il sito web
http://creativecommons.org/licenses/by-sa/3.0/ o spedisci una lettera
a Creative Commons, 171 Second Street, Suite 300, San Francisco,
California, 94105, USA.
\end{frame}

\end{document}
